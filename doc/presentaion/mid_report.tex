%%%%%%%%%%%%%%%%%%%%%%%%%%%%%%%%%%%%%%%%%%%%%%%%%%%%%%%%%%%%
%%  This Beamer template was created by Cameron Bracken.
%%  Anyone can freely use or modify it for any purpose
%%  without attribution.
%%
%%  Last Modified: January 9, 2009
%%

\documentclass[xcolor=x11names,compress]{beamer}

%% General document %%%%%%%%%%%%%%%%%%%%%%%%%%%%%%%%%%
\usepackage{graphicx}
\usepackage{tikz}
\usetikzlibrary{decorations.fractals}
%%%%%%%%%%%%%%%%%%%%%%%%%%%%%%%%%%%%%%%%%%%%%%%%%%%%%%


%% Beamer Layout %%%%%%%%%%%%%%%%%%%%%%%%%%%%%%%%%%
\useoutertheme[subsection=false,shadow]{miniframes}
\useinnertheme{default}
\usefonttheme{serif}
\usepackage{palatino}

\setbeamerfont{title like}{shape=\scshape}
\setbeamerfont{frametitle}{shape=\scshape}

\setbeamercolor*{lower separation line head}{bg=DeepSkyBlue4} 
\setbeamercolor*{normal text}{fg=black,bg=white} 
\setbeamercolor*{alerted text}{fg=red} 
\setbeamercolor*{example text}{fg=black} 
\setbeamercolor*{structure}{fg=black} 
 
\setbeamercolor*{palette tertiary}{fg=black,bg=black!10} 
\setbeamercolor*{palette quaternary}{fg=black,bg=black!10} 

\renewcommand{\(}{\begin{columns}}
\renewcommand{\)}{\end{columns}}
\newcommand{\<}[1]{\begin{column}{#1}}
\renewcommand{\>}{\end{column}}
%%%%%%%%%%%%%%%%%%%%%%%%%%%%%%%%%%%%%%%%%%%%%%%%%%




\begin{document}


%%%%%%%%%%%%%%%%%%%%%%%%%%%%%%%%%%%%%%%%%%%%%%%%%%%%%%
%%%%%%%%%%%%%%%%%%%%%%%%%%%%%%%%%%%%%%%%%%%%%%%%%%%%%%
\section{\scshape Project}
\begin{frame}
\title{Parallel Semi-Lagrangian Scheme}
\subtitle{For solving advection equations}
\author{
	Rongting Zhang, Yimin Zhong\\
	{\it Dept. Mathematics, University of Texas at Autin}\\
}
\date{
	\begin{tikzpicture}[decoration=Koch curve type 2] 
		\draw[DeepSkyBlue3] decorate{ decorate{ decorate{ (0,0) -- (4,0) }}}; 
	\end{tikzpicture}  
	\\
	\vspace{1cm}
	\today
}
\titlepage
\end{frame}

%%%%%%%%%%%%%%%%%%%%%%%%%%%%%%%%%%%%%%%%%%%%%%%%%%%%%%
%%%%%%%%%%%%%%%%%%%%%%%%%%%%%%%%%%%%%%%%%%%%%%%%%%%%%%


%%%%%%%%%%%%%%%%%%%%%%%%%%%%%%%%%%%%%%%%%%%%%%%%%%%%%%
%%%%%%%%%%%%%%%%%%%%%%%%%%%%%%%%%%%%%%%%%%%%%%%%%%%%%%
\section{\scshape Problem}
\subsection{Dynamical system}
\begin{frame}{Dynamical System in $\mathbb{R}^d$}
It is widely used in numerical weather prediction and various simulation.
\begin{eqnarray}
\frac{\mathrm{d}\mathbf{X}(t)}{\mathrm{d} t} = \mathbf{V}(\mathbf{X}(t),t)
\end{eqnarray}
or equivalently finding \emph{characteristic} of advection equation
\begin{eqnarray}
\frac{\partial f(\mathbf{x},t)}{\partial t} + \mathbf{V}(\mathbf{x},t)\cdot \nabla_x\; f = 0
\end{eqnarray}
\end{frame}

%%%%%%%%%%%%%%%%%%%%%%%%%%%%%%%%%%%%%%%%%%%%%%%%%%%%%%
%%%%%%%%%%%%%%%%%%%%%%%%%%%%%%%%%%%%%%%%%%%%%%%%%%%%%%

%%%%%%%%%%%%%%%%%%%%%%%%%%%%%%%%%%%%%%%%%%%%%%%%%%%%%%
%%%%%%%%%%%%%%%%%%%%%%%%%%%%%%%%%%%%%%%%%%%%%%%%%%%%%%

%%%%%%%%%%%%%%%%%%%%%%%%%%%%%%%%%%%%%%%%%%%%%%%%%%%%%%
%%%%%%%%%%%%%%%%%%%%%%%%%%%%%%%%%%%%%%%%%%%%%%%%%%%%%%
\section{\scshape Numerical Method}
\subsection{Semi-Lagrangian}
\begin{frame}
Method
\begin{itemize}
\item Semi-Lagrangian
\begin{eqnarray}
f(\mathbf{x}_m,t_n+\Delta t)=f(\mathbf{x}_m-2\mathbf{\alpha}_m,t_n-\Delta t)
\end{eqnarray}
\item Second order accuracy and iteration
\begin{eqnarray}
\mathbf{\alpha}_m=\Delta t\;\mathbf{V}(\mathbf{x_m}-\mathbf{\alpha}_m,t_n)
\end{eqnarray}
\end{itemize}
Pro
\begin{itemize}
\item No need to worry about CFL condition.
\end{itemize}
Con
\begin{itemize}
\item This scheme is at best second order accurate in approximating the backtracking point, if $\mathbf{V}$ is variable.
%\item if $\mathbf{V}$ is not divergence free, there may be oscillation around shock.
\end{itemize}
\end{frame}

\subsection{Semi-Lagrangian ENO/WENO}
\begin{frame}{Semi-Lagrangian ENO/WENO}
General Steps
\begin{itemize}
\item 1D problem:
\begin{itemize}
\item Backtracking in time.
\item Spacial interpolation using ENO/WENO.
\end{itemize}
\item Multi-D problem:\\
Strang splitting: decompose into severl 1D problems.
\end{itemize}
\end{frame}

\section{\scshape Numerical Complexity}
\subsection{Sequential}


\begin{frame}{Sequential Method}
\begin{itemize}
\item If $\mathbf{V}$ is constant vector, then it is fast to know the solution by simply interpolation and implement high order accuracy scheme by ENO/WENO.
\item To achieve $2k+1$ order accuracy in $\mathbf{R}^d$, with number of time steps as $N$ and mesh size $M$, by using WENO. Time complexity is $O(kNM^d)$. (Qiu \& Shu 2011)
=======
\item If $\mathbf{V}$ is constant vector, then it is fast to know the solution by simply interpolation, and easy to implement arbitrary accuracy scheme by ENO/WENO.
\item To achieve $2k+1$ order accuracy in $\mathbf{R}^d$, with time step as $N$ and mesh size $M$, by using WENO. Time complexity is $O(kNM^d)$.
\end{itemize}
\end{frame}


%%%%%%%%%%%%%%%%%%%%%%%%%%%%%%%%%%%%%%%%%%%%%%%%%%%%%%
%%%%%%%%%%%%%%%%%%%%%%%%%%%%%%%%%%%%%%%%%%%%%%%%%%%%%%
\subsection{Sequential}
\begin{frame}{Sequential Method}
\begin{itemize}
\item If $\mathbf{V}$ is variable and divergence free. Tracking back is no longer an easy job within high accuracy. We shall use some high order time integrator to implement this. 
\item For example, Semi-Lagrangian method with Runge-Kutta high order solver(time integrator) for back tracking. The time complexity is $O(M^d N\times T(SL))$ 
with second order time splitting and WENO. (Qiu \& Shu 2011)
\end{itemize}
\end{frame}

%%%%%%%%%%%%%%%%%%%%%%%%%%%%%%%%%%%%%%%%%%%%%%%%%%%%%%
%%%%%%%%%%%%%%%%%%%%%%%%%%%%%%%%%%%%%%%%%%%%%%%%%%%%%%
\subsection{Parallel}
\begin{frame}{Parallel Method}
Most of running time are consumed in Semi-Lagrangian back tracking and interpolation. And this can be parallelized perfectly, since back tracking with different end points $\mathbf{x}_m$ are independent process. Work-depth model time complexity here is $O(T(SL)\times N)$.
\end{frame}

\section{Milestones}
\begin{frame}{Milestiones}
\begin{itemize}
\item 1D Sequential Semi-Lagrangian ENO/WENO implementation. (April 12)
\item 3D Semi-Lagrangian ENO/WENO implementation. (April 18 )
\item 3D shared memory implementation. (April 25)
\item 3D distributed memory implementation. (May 9)
\end{itemize}
\end{frame}

\end{document}