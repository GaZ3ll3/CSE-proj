\documentclass[PROP,PDF]{prop} % use PDF command to enable PDFLaTeX driver
\usepackage{layout}

\usepackage{geometry}
\geometry{left=2.8cm,right=2.8cm,top=3.5cm,bottom=4cm,headheight=2.5cm,footskip=3.0cm}
\usepackage{amsmath}
\usepackage{amsfonts}
\usepackage{graphicx}
\usepackage{pdfpages}
\usepackage{listings}
\usepackage{hyperref}
\lstset{basicstyle=\footnotesize\ttfamily,breaklines=true, breakatwhitespace=true}
\DeclareGraphicsExtensions{.pdf,.png,.jpg}

\title{Semi-Lagrangian method on Octree}

\articletype{Project Proposal} % Research Article, Review Article, Communication, Erratum

\def\checkbox{\indent\text{\rlap{$\checkmark$}}\square}

\def\uncheckbox{\indent\square}



\author{Rongting~Zhang\inst{1}\email{rzhang@math.utexas.edu},
        Yimin~Zhong\inst{1}\email{yzhong@math.utexas.edu}
       }

\shortauthor{R. Zhang, Y. Zhong}

\institute{\inst{1}
           Department of Mathematics, University of Texas at Austin, 78712, Austin, U.S.A
          }

% \abstract{An abstract should accompany every article. It should be a brief summary of significant results of the paper. An abstract should give concise information about the content of the core idea of your paper. It should be informative and not only present the general scope of the paper but also indicate the main results and conclusions.\\
% The abstract should not exceed 200 words. It should not contain literature citations, allusions to the tables, tables, figures or illustrations. All nonstandard symbols and abbreviations should be defined. In combination with the title and keywords, an abstract is an indicator of the content of the paper.
% }

% \keywords{Keyword 1 \*\ Keyword 2 \*\ Keyword 3}

% \msc{XXXXX, YYYYY}

\begin{document}
\maketitle
%\baselinestretch{2}
\section{Introduction}
The octree data structure\cite{OM1}\cite{OM2} with adaptive mesh refinement proves to be a flexible and allows accurate and efficient tracking of flow features. On the other hand, semi-Lagrangian method\cite{JS}\cite{SL} enables the choice of a larger time steps in the simulation of fluids which would be crucial for an adaptive mesh approach. Popinet\cite{GE} and Gibou\cite{ST} have previously worked on solving Euler equation on octree data structure. In this project, we try to implement a parallelized semi-Lagrangian method on octree data structure for a simple transport equation and Euler equation if time permits. 
\section{Problem}
Solve transport equation or incompressible Euler equation on \emph{Octree} Data structure with cell centered discretization and semi-Lagrangian scheme.
\subsection{Non adaptive mesh}
\begin{enumerate}
\item Regular Cartesian grid\\
A three time level semi-Lagrangian method will be adopted. See section ...
\item Time-invariant octree\\
We will adopt the implementation of 2:1 balance octree.  
\end{enumerate}

\subsection{Adaptive mesh}
\begin{enumerate}
\item Refine criterion\\
An attractive option is to use Richardson extrapolation. Simple criterion based on the norm of the local vorticity vector can also be considered, e.g. $\frac{h||\nabla \times U||}{\max||U||}>\tau$. \cite{GE}
\item Balance octree after refining\\
Refinement of a single cell may cause dismatch of level with neighboring cells. Thus a rebalancing of the octree is required. \cite{OM1}\cite{OM2} 
\item The level of refinement\\
After each refinement and re-balancing we will need to do one more cycle of calculation. This will be expensive if a few cells need multilevel refinement.\\ 
Possible solution:\\
1. In Gerris they would do at most 1-level refinement each time-step.\\
2. Maximum refinement can be taken as parameter in the program. 
\end{enumerate}
 

\section{Semi-Lagragian}
Use time level temporal discretization and update each grid point using
\begin{equation}
\frac{F(x_m,t_n+\Delta t)-F(x_m-2\alpha_m,t_n-\Delta t)}{2\Delta t}=0
\end{equation} 
$F(x_m-2\alpha_m,t_n-\Delta t)$ second order interpolation 
Find $\alpha_m$ by
\begin{equation}
\alpha_m=\Delta t\; U(x_m-\alpha_m,t_n)
\end{equation}
where $U(x,t)$ is the velocity field.
\subsection{Iteration}
Use fix point iteration as following,
\begin{eqnarray}
\alpha_{m}^{k+1} = \Delta t\; U (x_m - \alpha_m^{k},t_n)
\end{eqnarray}
The condition for above iteration to converge is
\begin{eqnarray}
\Delta t \; \max_x |\nabla_x U| \le 1
\end{eqnarray}
\subsection{Computing Concerns}
\begin{enumerate}
\item Interpolation\\
When neighboring cells are of different level, we will need to do interpolation to calculate gradient flux on the interface of cells in Euler equation. Here we plan to adopt the strategy introduced in Gerris.
\item Boundary condition\\
For Euler equation, we shall use no-flow condition.
\end{enumerate}

\section{Complexity for work-depth model}

\subsection{Semi-Lagrangian}
Suppose the tolerance of each iteration is $tol$, then steps for convergence is $\Omega(|\log(tol)|)$. Since each iteration is independent, the time complexity is $\Omega(1/\Delta t |\log(tol)|)$, work complexity is $\Omega(N^d/\Delta t |\log(tol)|)$. 
\subsection{Octree}
The complexity here depends on specific implementations. \cite{OM1},\cite{OM2}.

And the octree related operations required:
\begin{enumerate}
\item Time invariant:
\begin{enumerate}
\item Generation of octree.
\item Fetch data from neighboring cells.
\end{enumerate}
\item Extra operations for adaptive:
\begin{enumerate}
\item (Check refinement/coarsening criterion)
\item Refinement/coarsening of octree
\end{enumerate}
\end{enumerate}


\section{Milestone}
\begin{description}
\item $\checkbox$ 2-D and 3-D regular grid semi-Lagrangian
\item $\uncheckbox$ Parallelized semi-Lagrangian on regular grid -- (1 week)
\item $\uncheckbox$ Semi-Lagrangian on time-invariant Octree -- (1 week)
\item $\uncheckbox$ Semi-Lagrangian on adaptive Octree -- (2 week)
\end{description}

\section{Possible issues}
\begin{description}
\item $\checkbox$ Octree Package interface
\item $\uncheckbox$ Overhead of the adaptive octree method
\end{description}

\section{Bitbucket Repository}
The project is hosted on \href{https://bitbucket.org/nlmd/cse-proj}{https://bitbucket.org/nlmd/cse-proj}.
\begin{thebibliography}{6}

\bibitem{GE} St\'{e}phane Popinet, Gerris: A Tree-Based Adaptive Solver For The Incompressible Euler Equations In Complex Geometries 2003, 190, 572--600

\bibitem{OM1} Carsten Burstedde, Lucsa C. Wilcox, Omar Ghattas,
 p4est: Scalable Algorithms for Parallel Adaptive Mesh Refinement on Forests of Octrees., SIAM J. SCI.COMPUT, 2011 , 33 , 1103--1133

\bibitem{OM2} Tobin Isaac, Carsten Burstedde, and Omar Ghattas,
Low-Cost Parallel Algorithms for 2:1 Octree Balance.
Published in Proceedings of the 26th IEEE International Parallel \& Distributed Processing Symposium, 2012. 

\bibitem{JS} Stable Fluids, Jos Stam, SIGGRAPH 99 Conference Proceedings, Annual Conference Series, August 1999, 121-128

\bibitem{SL} Semi-Lagrangian Integration Schemes for Atmospheric Models -- A Review, Andrew Staniforth and Jean C\^{o}t\'{e}, 1991, 19, 2206 --2223

\bibitem{ST}Losasso, Frank, Fr\'{e}d\'{e}ric Gibou, and Ron Fedkiw. "Simulating water and smoke with an octree data structure." ACM Transactions on Graphics (TOG). Vol. 23. No. 3. ACM, 2004.



\end{thebibliography}

% \begin{table}
% \def~{\phantom0}
% \catcode`\@=13
% \def@{\phantom.}
% \caption{Some caption text.\label{tab-01}}
% \medskip
% \begin{center}
% \begin{tabular}{l|ccc}\hline
% \multicolumn1l{\it Some title}\\\hline\hline
% row 1, column 1         &   row 1, column 2  \\
% row 2, column 1    &   row 2, column 2   \\
% row 3, column 1    &   row 3, column 2   \\
% \hline
% \multicolumn1l{\it Another title} & Value 1 &   Value 2 &   Value 3 \\
% \hline\hline
% row 1                    & ~130 & 30 & 30 \\
% row 2                    & 1025 & ~1 & 15 \\
% row 3                   & ~100 & ~1 & 10 \\
% row 4        & 2925 & ~1 & ~4 \\
% row 5            & 2950 & ~1 & ~2 \\\hline
% \end{tabular}
% \end{center}
% \end{table}


% \begin{figure}
% \caption{The graph of $y=x\,sin x$ in the interval [-50,50], created by wxMaxima 0.7.4. \label{fig_abc}}
% %\vskip3cm
% \includegraphics{xsinx-49.png}
% \end{figure}






\end{document}
